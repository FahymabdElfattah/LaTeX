
\addcontentsline{toc}{section}{\textcolor{cyan}{\textbf{Introduction}}}
\begin{flushleft}
\section*{\textcolor{cyan}{Introduction}}
 Le livre \cite{texbook} est une référence importante pour les utilisateurs de TeX.
 La méthode AGILE est une méthode de gestion de projet. Il existe en réalité plusieurs méthodes qui ont toutes un point commun : elles découlent toutes du manifeste Agile. Édité en 2001, le manifeste AGILE a été développé par plusieurs développeurs de logiciels. Son but : améliorer leur processus et réduire leur taux d'échec. Pour cela, ils placent le client au cœur du projet et ils s'adaptent tout le long du fil du projet. C'est donc une toute nouvelle façon de voir les choses et d'aborder le développement d'un produit, d'un service ou d'un projet. Depuis, les méthodes qui s'inscrivent dans la philosophie de ce manifeste sont appelées méthodes agiles certaines parlent également de lean agile.\newline	Une méthodologie agile prévoit la fixation d’objectifs à court terme utilisé dans la gestion de projets. Alors que la méthode traditionnelle prévoit la planification totale du projet avant même la phase développement. Le projet n'est ainsi fragmenté en plusieurs sous-parties où les équipes de développement qui en a la charge doivent atteindre progressivement en ajustant si nécessaire les objectifs pour répondre le plus possible aux attentes du client. On parle alors de sprints. Chaque spring ayant pour objectif de clôturer une brique du projet. Les méthodes agiles permettent de renforcer les relations entre les membres d’une équipe, mais aussi entre l’équipe et le client. C’est pour cette raison que la flexibilité et la souplesse sont deux piliers fortes dans la méthode AGILE. On parle d’un mode itératif qui permet de délivrer des prestations avec efficacité et réactivité le tout en respectant des plannings exigeants. Cela permet de gérer ses projets de façon performante.\\	\textbf{Application de la méthode agile à notre projet :}\\	Dans ce qui vient on va appliquer la méthode agile scrum sur un projet que nous avons développé dans L'élément de module projet de développement ou projet de fin d'année.
	
Ce projet consisté à réaliser un prototype d'une maison intelligente commander par une application android on se base sur la communication Bluetooth.
\end{flushleft}
\bibliographystyle{plain}
\newpage
	