
\addcontentsline{toc}{section}{\textcolor{cyan}{Conclusion}}
\begin{flushleft}
	\section*{\textcolor{cyan}{Conclusion}}
	La méthode AGILE est une méthode de gestion de projet. Il existe en réalité plusieurs méthodes qui ont toutes un point commun : elles découlent toutes du manifeste Agile. Édité en 2001, le manifeste AGILE a été développé par plusieurs développeurs de logiciels. Son but : améliorer leur processus et réduire leur taux d'échec. Pour cela, ils placent le client au cœur du projet et ils s'adaptent tout le long du fil du projet. C'est donc une toute nouvelle façon de voir les choses et d'aborder le développement d'un produit, d'un service ou d'un projet. Depuis, les méthodes qui s'inscrivent dans la philosophie de ce manifeste sont appelées méthodes agiles certaines parlent également de lean agile.\newline	Une méthodologie agile prévoit la fixation d’objectifs à court terme utilisé dans la gestion de projets. Alors que la méthode  .
	Concilter \cite{Doe2018}
\end{flushleft}

\newpage
