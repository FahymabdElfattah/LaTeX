\begin{tikzpicture}[remember picture,overlay]
	\coordinate(1)at ([shift={(225:\size)}]current page.north east);
	\coordinate(2)at ([shift={(-45:\size)}]current page.north west);
	\coordinate(3)at ([shift={(45:\size)}]current page.south west);
	\coordinate(4)at ([shift={(135:\size)}]current page.south east);
	\foreach \i in {0,1,...,60}{\foreach \j/\k in {1/4,2/3}{
			\path(\j)--node[pos=\i/60]{\shape{0.1}}(\k);
	}}
	\foreach \i in {1,2,...,45}{\foreach \j/\k in {1/2,4/3}{
			\path(\j)--node[pos=\i/45]{\shape{0.1}}(\k);
	}}
\end{tikzpicture}
{\includegraphics[scale =0.5]{chapitres/images/inpt.png}} \hfill
\hspace{-2cm}
\begin{minipage}[c]{10cm}
	\centering \bf
	Université Moulay Ismail \\
	Faculté des sciences Meknès \\
	Département matmathématiques\\
	et Informatique
\end{minipage}
\hfill
\hspace{-2cm}
{\includegraphics[scale =0.4 ]{chapitres/images/anrt.png}
	\vglue5mm
	\centering\vglue3cm\sl
	\underline{\bf\Large MEMOIRE DE FIN D'ÉTUDE }
	\vglue5mm
	\vspace{2cm}
	\begin{center}
		
		\vskip3mm
		\titlebox{%
			Titre de projet\\........\\.......
		}
	\end{center}
	\vglue5mm
	\vskip5mm
	\vspace{0.2cm}
	\vfill
	\begin{minipage}[t]{10cm}
		\underline{\bf  Présenté par:}\\[3mm]\bf
		\begin{itemize}
			\item FAHYM Abd Elfattah
			\item XXXXXX
			\item XXXXXXX
		\end{itemize}
	\end{minipage}
	\hfill
	\begin{minipage}[t]{5cm}
		\underline{\bf Enquadré par:}\\[3mm]\bf
		\hfill Pr.XXXXXXX
	\end{minipage}
	\vfill
	\vfill
	\centering\bf
	Promotion: 2020 /2021
	\newpage